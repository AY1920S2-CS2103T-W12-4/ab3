\documentclass[11pt,british]{article}

\usepackage[a4paper,margin=2.5cm]{geometry}
\usepackage[utf8]{inputenc}
\usepackage{inconsolata}
\usepackage{MnSymbol}
\usepackage[
	minionint,
	lf,
	mathtabular,
	loosequotes,
	swash,
	opticals]{MinionPro}
\usepackage{amsmath}%, amsfonts, amssymb}
% \usepackage{wasysym}
\usepackage[useregional,calc]{datetime2}
\usepackage[super]{nth}
\input{glyphtounicode}
\pdfgentounicode=1
\pdfminorversion=7

\usepackage[
	arc-separator = \,,
	retain-explicit-plus,
	detect-weight=true,
	detect-family=true,
	separate-uncertainty=true,
	multi-part-units=brackets]{siunitx}
\usepackage[version=4]{mhchem}
\usepackage[ISO]{diffcoeff}
\usepackage{bm}
\usepackage{esvect}

\usepackage{enumitem}
\usepackage{parskip}
\usepackage{multicol}
\usepackage{titlesec}
\usepackage{microtype}
\usepackage{bigfoot}

\usepackage{tabularx}
\usepackage{booktabs}

%\usepackage{listings}
\usepackage[usenames,dvipsnames]{xcolor}
\usepackage{pgfplots,pgfplotstable}
\usepackage[skins,theorems]{tcolorbox}
\usepackage{graphicx}
\usepackage{epstopdf}
\usepackage[labelfont={small,bf},font={small}]{caption}
\usepackage{subcaption}
\usepackage{float}
\tcbset{shield externalize,highlight math style={enhanced,
  colframe=red,colback=white,arc=0pt,boxrule=1pt}}
\pgfplotsset{compat=newest}
\usetikzlibrary{
	shapes,shapes.misc,
	arrows,arrows.meta,
	calc,positioning,
	decorations.pathreplacing,decorations.markings,
	decorations.text,calligraphy,
	pgfplots.dateplot,
	optics,external,
	circuits.ee.IEC
}
\tikzset{
	on each segment/.style={
		decorate,
		decoration={
			show path construction,
			moveto code={},
			lineto code={
				\path [#1]
				(\tikzinputsegmentfirst) -- (\tikzinputsegmentlast);
			},
			curveto code={
				\path [#1] (\tikzinputsegmentfirst)
				.. controls
				(\tikzinputsegmentsupporta) and (\tikzinputsegmentsupportb)
				..
				(\tikzinputsegmentlast);
			},
			closepath code={
				\path [#1]
				(\tikzinputsegmentfirst) -- (\tikzinputsegmentlast);
			},
		},
	},
	mid arrow/.style={postaction={decorate,decoration={
				markings,
				mark=at position .5 with {\arrow[#1]{Stealth}}
	}}},
	>=Stealth
}
\newcommand*\circled[1]{\tikz[baseline=(char.base)]{%
		\node[shape=circle, draw, minimum size=1.25em, inner sep=0pt, thick] (char) {#1};}}
\tikzexternalize[prefix=figures/]
\captionsetup{width=0.6\textwidth}

\usepackage[nottoc,numbib]{tocbibind}
\usepackage[
	backend=biber,
	language=british,
    backref=true,
	style=verbose-ieee,
    bibstyle=numeric,
    citestyle=numeric,
    sorting=none
]{biblatex}

%\addbibresource{'hall_effect_report.bib'}

\DefineBibliographyStrings{english}{%
    backrefpage = {page},% originally "cited on page"
    backrefpages = {pages},% originally "cited on pages"
}
\DeclareCiteCommand{\supercite}[\mkbibsuperscript]
{\iffieldundef{prenote}
    {}
    {\BibliographyWarning{Ignoring prenote argument}}%
    \iffieldundef{postnote}
    {}
    {\BibliographyWarning{Ignoring postnote argument}}}
{\usebibmacro{citeindex}%
    \bibopenbracket\usebibmacro{cite}\bibclosebracket}
{\supercitedelim}
{}
\let\cite=\supercite

\usepackage{hyperref}
\usepackage[capitalise,noabbrev,nameinlink]{cleveref}
\crefdefaultlabelformat{#2\textbf{#1}#3}
\creflabelformat{equation}{#2\textbf{(#1)}#3}
\crefname{equation}{\textbf{Equation}}{\textbf{Equations}}
\Crefname{equation}{\textbf{Equation}}{\textbf{Equations}}
\crefname{figure}{\textbf{Figure}}{\textbf{Figures}}
\Crefname{figure}{\textbf{Figure}}{\textbf{Figures}}
\crefname{table}{\textbf{Table}}{\textbf{Tables}}
\Crefname{table}{\textbf{Table}}{\textbf{Tables}}
\crefname{appendix}{\textbf{Appendix}}{\textbf{Appendices}}
\Crefname{appendix}{\textbf{Appendix}}{\textbf{Appendices}}
\crefname{section}{\textbf{\S}}{\textbf{\S}}
\Crefname{section}{\textbf{\S}}{\textbf{\S}}
\crefname{algorithm}{\textbf{Algorithm}}{\textbf{Algorithms}}
\Crefname{algorithm}{\textbf{Algorithm}}{\textbf{Algorithms}}

\hypersetup{
	unicode 	 = true,
	colorlinks   = true,           %Colours links instead of ugly boxes
	urlcolor     = PineGreen,     %Colour for external hyperlinks
	linkcolor    = NavyBlue,         %Colour of internal links
	citecolor    = Magenta,   %Colour of citations
	linktocpage  = true
}

\DTMlangsetup*{ord=raise}

\DeclareMathOperator{\sinc}{sinc}

\setlength{\jot}{10pt}

\setlist[itemize]{left=0pt}
\setlist[enumerate,1]{left=0pt}

%\lstdefinestyle{code}{
%	backgroundcolor=\color{backcolour},
%	commentstyle=\color{codegreen},
%	keywordstyle=\color{magenta},
%	numberstyle=\tiny\color{codegray},
%	stringstyle=\color{codepurple},
%	basicstyle=\ttfamily\footnotesize,
%	breakatwhitespace=false,
%	breaklines=true,
%	captionpos=b,
%	keepspaces=true,
%	numbers=left,
%	numbersep=5pt,
%	showspaces=false,
%	showstringspaces=false,
%	showtabs=false,
%	tabsize=2
%}

%\lstset{style=code}


\title{\textbf{CS2103T} \\ Team Project: CookBuddy}
\author{Sharadh Rajaraman \\ \textbf{A0189906L}}
\begin{document}
\maketitle

\section{Overview}\label{overview}

\href{}{} is a desktop recipe manager for students living in
on-campus accommodation, and who enjoy cooking, and are also familiar
with the command line.

Users may interact with CookBuddy through a Command Line Interface
(CLI), and it has a Graphical User Interface (GUI) created with JavaFX.

CookBuddy is written in Java, and has about \num{10000} lines of code.

\section{Summary of contributions}\label{summaryofcontributions}

\begin{itemize}
    \item
    \textbf{Major enhancement}: added \textbf{Image reading, processing and writing}.
    \begin{itemize}
        \item \textbf{What it does:} Allows users to add images to their recipes, by specifying a relative or absolute
        filepath. Images are saved in a data folder local to CookBuddy, and are named with a
        UID (unique identifier). If an image cannot be found at the specified path, a placeholder is used instead.
        \item
        Justification: Users should be able to see what the final dish looks
        like, and an image goes a very long way in showing that. Furthermore, images are a prominent feature of most
        competing recipe managers, and this is a key feature that should have been implemented.
        \item
        Highlights: This enhancement affects existing commands and commands
        to be added in future. It required an in-depth analysis of design
        alternatives. The implementation too was challenging as it required
        changes to existing commands.
        \item
        Credits: \emph{\{mention here if you reused any code/ideas from
            elsewhere or if a third-party library is heavily used in the feature
            so that a reader can make a more accurate judgement of how much
            effort went into the feature\}}
    \end{itemize}
    \item
    \textbf{Minor enhancement}: added a history command that allows the
    user to navigate to previous commands using up/down keys.
    \item
    \textbf{Code contributed}:
    \item
    \textbf{Other contributions}:
    \begin{itemize}
        \item
        Project management:

        \begin{itemize}
            \item
            Managed releases \texttt{v1.3} - \texttt{v1.5rc} (3 releases) on
            GitHub
        \end{itemize}
        \item
        Enhancements to existing features:
        \item
        Documentation:

        \begin{itemize}
            \item
            Did cosmetic tweaks to existing contents of the User Guide:
        \end{itemize}
        \item
        Community:

        \begin{itemize}
            \item
            PRs reviewed (with non-trivial review comments)
        \end{itemize}
        \item
        Tools:

        \begin{itemize}
            \item
            Integrated a third party library (Natty) to the project
            \item
            Integrated a new Github plugin (CircleCI) to the team repo
        \end{itemize}
    \end{itemize}
\end{itemize}
\end{document}
